%%%%%%%%%%%%%%%%%%%%%%%%%%%%%%%%%%%%%%%%%
% Friggeri Resume/CV
% XeLaTeX Template
% Version 1.2 (3/5/15)
%
% This template has been downloaded from:
% http://www.LaTeXTemplates.com
%
% Original author:
% Adrien Friggeri (adrien@friggeri.net)
% https://github.com/afriggeri/CV
%
% License:
% CC BY-NC-SA 3.0 (http://creativecommons.org/licenses/by-nc-sa/3.0/)
%
% Important notes:
% This template needs to be compiled with XeLaTeX and the bibliography, if used,
% needs to be compiled with biber rather than bibtex.
%
%%%%%%%%%%%%%%%%%%%%%%%%%%%%%%%%%%%%%%%%%

\documentclass[a4paper]{friggeri-cv} % Add 'print' as an option into the square bracket to remove colors from this template for printing
\addbibresource{bibliography.bib} % Specify the bibliography file to include publications

\begin{document}
\header{Guilherme}{Sousa}{En recherche de CDI en dévelopement logiciel pour systèmes embarqués} % Your name and current job title/field

%----------------------------------------------------------------------------------------
%	SIDEBAR SECTION
%----------------------------------------------------------------------------------------

\begin{aside} % In the aside, each new line forces a line break
\section{contact}
Résidence 6 Apt. 6126, Av Edouard Belin nº4
31400 Toulouse
0781726068
\href{mailto:guilherme.sousa1994@gmail.com}{guilherme.sousa1994\newline @gmail.com}
\href{https://www.linkedin.com/in/guilhermegsousa}{LinkedIn://guilhermegsousa}
\href{https://github.com/GuilhermeGSousa}{GitHub://GuilhermeGSousa}
\section{langages}
Langues maternelles Portugais et Français 
Anglais C2,
Espagnol B1,
Niveau A1 d'Allemand
\section{code}
Matlab \& Simulink
C, C++
Python
JAVA
\section{logiciels}
Linux, ROS
Eclipse, Qt
LateX, Office
SolidEdge/SolidWorks
OpenCV, Unity
\end{aside}

%----------------------------------------------------------------------------------------
%	EDUCATION SECTION
%----------------------------------------------------------------------------------------

\section{éducation}

\begin{entrylist}

%------------------------------------------------

\entry
{Présent}
{Master {\normalfont en Ingénierie Aérospatiale}}
{Instituto Superior Técnico, Portugal}
{}%Write Thesis Theme HERE!


\entry
{Fev-Mai 2015}
{ERASMUS, Master {\normalfont en Systèmes Méchatroniques} }
{IPSA, Paris}
{}
%------------------------------------------------

\entry
{2012-2015}
{BAC+3 {\normalfont en Ingénierie Aérospatiale}}
{Instituto Superior Técnico, Portugal}
{Spécialisation en Avionique}

%------------------------------------------------

\end{entrylist}

%----------------------------------------------------------------------------------------
%	WORK EXPERIENCE SECTION
%----------------------------------------------------------------------------------------

\section{expérience}


\subsection{Part-Time et Stages}

\begin{entrylist}
\entry
{Présent}
{ENAC (stage fin d'études)}
{Toulouse, France}
{Développement au MAIAA (département de Mathématiques Appliquées, Informatique et Automation pour l'Aérien) d'un contrôleur automatique adaptatif pour avions commerciaux en utilisant des réseaux de neurones}

\entry
{Présent}
{Google Summer of Code}
{Toulouse, France}
{Actuellement en phase de préselection pour contribuer pour le projet opensource ArduPilot, dans le cadre du Google Summer of Code 2017, avec un système de rédondance pour drones en cas de panne de moteur}


\entry
{Février-Mai 2016}
{IPSA Space Systems}
{Paris, France}
{Développement d'un système de traitement et filtrage de données en temps-réel pour le projet de la fusée Jericho, de l'association étudiante ISS avec un partenariat du CNES (Centre National d'Études Spatiales).}


\entry
{Août 2015}
{ProDrone}
{Lisbonne, Portugal}
{\emph{Stage d'été} \\
Responsable pour l'implémentation d'un algorithme de contrôle d'un drone autonome, utilisé dans le domaine de l'inspection de turbines éoliennes}

%------------------------------------------------

%------------------------------------------------

\end{entrylist}

%----------------------------------------------------------------------------------------
%	AWARDS SECTION
%----------------------------------------------------------------------------------------

\section{projets et distinctions}

\begin{entrylist}

%------------------------------------------------


\entry
{2016}
{Implémentation du système TCAS}
{IST}
{Implémentation d'un système TCAS en C++ utilisant la bibliothèque graphique Qt pour simulation de colisions en espace aérien, et le protocole UDP pour communication entre les aéronefs simulés}
\entry
{2016}
{Projet Drone terrestre autonome}
{IPSA}
{Conception sur CATIA d'un drone terrestre et développement de son algorithme de contrôle pour un projet universitaire ERASMUS}
\entry
{2015}
{Algorithme IMM pour systèmes ATC}
{IST}
{Implémentation d'un algorithme IMM pour deux filtres Kalman sur Simulink pour filtrer des données d'un simulateur de radar}
\entry
{2014}
{Jeu Mobile Publié}
{App Store}
{Développeur d'une application publiée nommée "Blockalicious!" en JAVA sur la plateforme Google Play Store.
\href{https://play.google.com/store/apps/details?id=com.guiero.blockaliciousfinal}{\textbf{Lien Google Play}}}
%------------------------------------------------
\end{entrylist}

%----------------------------------------------------------------------------------------
%	COMMUNICATION SKILLS SECTION
%----------------------------------------------------------------------------------------

%\section{communication skills}

%----------------------------------------------------------------------------------------
%	INTERESTS SECTION
%----------------------------------------------------------------------------------------

\section{intérêts}

Ancien joueur de l'équipe nationale portugaise de rugby (under 19), actuellement nº8 de l'équipe de rugby de l'ENAC

%----------------------------------------------------------------------------------------

\end{document}