%%%%%%%%%%%%%%%%%%%%%%%%%%%%%%%%%%%%%%%%%
% Friggeri Resume/CV
% XeLaTeX Template
% Version 1.2 (3/5/15)
%
% This template has been downloaded from:
% http://www.LaTeXTemplates.com
%
% Original author:
% Adrien Friggeri (adrien@friggeri.net)
% https://github.com/afriggeri/CV
%
% License:
% CC BY-NC-SA 3.0 (http://creativecommons.org/licenses/by-nc-sa/3.0/)
%
% Important notes:
% This template needs to be compiled with XeLaTeX and the bibliography, if used,
% needs to be compiled with biber rather than bibtex.
%
%%%%%%%%%%%%%%%%%%%%%%%%%%%%%%%%%%%%%%%%%

\documentclass[]{friggeri-cv} % Add 'print' as an option into the square bracket to remove colors from this template for printing
\addbibresource{bibliography.bib} % Specify the bibliography file to include publications

\begin{document}
\header{Guilherme}{Sousa}{} % Your name and current job title/field

%----------------------------------------------------------------------------------------
%	SIDEBAR SECTION
%----------------------------------------------------------------------------------------

\begin{aside} % In the aside, each new line forces a line break
\section{contact}
Rue de Nossa Senhora Da Conceição, 21
Alcoitão
2645-151 Alcabideche\newline
+351 91 280 36 59
\href{mailto:guilherme.sousa1994@gmail.com}{guilherme.sousa1994\newline @gmail.com}
\href{https://www.linkedin.com/in/guilhermegsousa}{LinkedIn://guilhermegsousa}
\href{https://github.com/GuilhermeGSousa}{GitHub://GuilhermeGSousa}
\section{langages}
Langues maternelles Portugais et Français 
Anglais C2,
Espagnol B1,
Niveau A1 d'Allemand
\section{code}
C/C++/C\#
Python,
JAVA \& Matlab/ Simulink/StateFlow
\section{logiciels}
SolidEdge/SolidWorks,
LateX, Office, Qt, Unity,
Linux \& ROS
\end{aside}

%----------------------------------------------------------------------------------------
%	EDUCATION SECTION
%----------------------------------------------------------------------------------------

\section{éducation}

\begin{entrylist}

%------------------------------------------------

\entry
{2015}
{Master {\normalfont en Ingénierie Aérospatiale}}
{Instituto Superior Técnico, Portugal}
{\emph{} \\ }%Write Thesis Theme HERE!


\entry
{2015}
{ERASMUS, Master {\normalfont en Systèmes Méchatroniques} }
{IPSA, Paris}
{\emph{} \\ }
%------------------------------------------------

\entry
{2012-\\-Présent}
{BAC+3 {\normalfont en Ingénierie Aérospatiale}}
{Instituto Superior Técnico, Portugal}
{Spécialisation en Avionique}

%------------------------------------------------

\end{entrylist}

%----------------------------------------------------------------------------------------
%	WORK EXPERIENCE SECTION
%----------------------------------------------------------------------------------------

\section{expérience}


\subsection{Part-Time et Stages}

\begin{entrylist}
\entry
{Présent}
{ENAC}
{Toulouse, France}
{Développement au MAIAA (département de Matématiques Apliquées, Informatique et Automation pour l'Aérien) d'un contrôleur automatique adaptatif pour avions commerciaux en utilisant des réseaux de neurones}



\entry
{Février\\-Mai 2016}
{IPSA Space Systems}
{Paris, France}
{Développement d'un système de traitement et filtrage de données en temps-réel pour le projet de la fusée Jericho, de l'association étudiante ISS avec un partenariat du CNES (Centre National d'Études Spatiales).}


\entry
{Août 2015}
{ProDrone}
{Lisbonne, Portugal}
{\emph{Stage d'été} \\
Responsable pour l'implémentation d'un algorithme de contrôle d'un drone autonome, utilisé dans le domaine de l'inspection de turbines éoliennes}

%------------------------------------------------

%------------------------------------------------

\end{entrylist}

%----------------------------------------------------------------------------------------
%	AWARDS SECTION
%----------------------------------------------------------------------------------------

\section{projets et distinctions}

\begin{entrylist}

%------------------------------------------------


\entry
{2016}
{Implémentation du système TCAS}
{IST}
{Implémentation d'un système TCAS en C++ utilisant la bibliothèque graphique Qt pour simulation de colisions en espace aérien, et le protocole UDP pour communication entre les aéronefs simulés}
\entry
{2016}
{Projet Drone terrestre autonome}
{IPSA}
{Conception sur CATIA d'un drone terrestre imprimé en 3D et développement de son algorithme de contrôle pour un projet universitaire pendant mon programme ERASMUS}
\entry
{2015}
{Algorithme IMM pour systèmes ATC}
{IST}
{Implémentation d'un algorithme IMM pour deux filtres Kalman sur Simulink pour filtrer des données d'un simulateur de radar}
\entry
{2014}
{Jeu Mobile Publié}
{App Store}
{Développeur d'une application publiée nommée "Blockalicious!" en JAVA utilisant OpenGL sur la plateforme Google Play Store.
\href{https://play.google.com/store/apps/details?id=com.guiero.blockaliciousfinal}{\textbf{Lien Google Play}}}
%------------------------------------------------
\end{entrylist}

%----------------------------------------------------------------------------------------
%	COMMUNICATION SKILLS SECTION
%----------------------------------------------------------------------------------------

%\section{communication skills}

%----------------------------------------------------------------------------------------
%	INTERESTS SECTION
%----------------------------------------------------------------------------------------



%----------------------------------------------------------------------------------------

\end{document}